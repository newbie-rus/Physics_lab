\documentclass[a4paper, 12pt]{article}

\usepackage{wrapfig}
\usepackage{graphicx}
\usepackage{mathtext}
\usepackage{amsmath}
\usepackage{siunitx} % Required for alignment
\usepackage{multirow}
\usepackage{rotating}
\usepackage{float}

\usepackage[T1,T2A]{fontenc}
\usepackage[russian]{babel}

\graphicspath{{pictures/}}

\title{\begin{center}Лабораторная работа №1.1.4\end{center}
Измерение интенсивности радиационного фона}
\author{Шляпин И.С.}
\date{\today}

\begin{document}
    \pagenumbering{gobble}
    \maketitle
    \newpage
    \pagenumbering{arabic}

    \section{Введение}
    \textbf{Цель работы:}
    \begin{itemize}
        \item Применить методы обработки экспериментальных данных для изучения статистических закономерностей при измерении интенсивности радиационного фона
    \end{itemize}

    \vspace{1cm}

    \textbf{В работе используются: }
    \begin{itemize}
        \item счетчик Гейгера-Мюллера
        \item блок питания
        \item компьютер с интерфейсом связи со счетчиком

    \end{itemize}

    \section{Ход работы}
    \paragraph{}
    Проведем измерение используя интерфейс компьютера. Приведем данные в таблицу и начнем  обработку. Разбивая данные для 20с по парам и просуммировав пары получим данные для 40с.
    \paragraph{}
    Проверим связь $\sigma_{отд}\approx \sqrt{\bar{n}}$. Индекс 1 для 10с, 2 для 40с
    \[n_{общ} = \sum n_i = 5223\]
    \[\bar{n}_1 = \frac{n_{общ}}{N_1} = 13.0575\]
    \[\bar{n}_2 = \frac{n_{общ}}{N_2} = 52.23\]
    \[\sigma_{1}=\sqrt{\frac{1}{N_1} \sum_{i=1}^{N_1} (n_i - \bar{n}_i)^2} \approx 3.41\]
    \[\sigma_{2}=\sqrt{\frac{1}{N_2} \sum_{i=1}^{N_2} (n_i - \bar{n}_i)^2} \approx 7.38\]
    \newpage

    \[\sqrt{\bar{n}_1} =3.61 \approx 3.41 = \sigma_1\]
    \[\sqrt{\bar{n}_2} =7.22 \approx 7.38 = \sigma_2\]
    \paragraph{}
    Как видим связь между среднеквадратическим отклонением и среднем значении есть $(\sigma \approx \sqrt{\bar{n}})$.
    Теперь определим долю случаев в пределах $\pm\sigma$ и $\pm2\sigma$.

    \begin{table}[H]
    \begin{center}
    \begin{tabular}{|c|c|c|c|}\hline
    \multicolumn{4}{|c|}{$t=10с$}\\\hline
    Предел & Число случаев & Доля случаев & Теоретическая оценка\\\hline
    $\pm \sigma_1 = \pm 3.41$ & 271 & 67.5\% & 66\% \\
    $\pm 2\sigma_1 = \pm 6.82$ & 381 & 95\% & 95\% \\\hline
    \multicolumn{4}{c}{}\\\hline
    \multicolumn{4}{|c|}{$t=40с$}\\\hline
    Предел & Число случаев & Доля случаев & Теоретическая оценка\\\hline
    $\pm \sigma_2 = \pm 7.38$ & 71 & 71\% & 68\% \\
    $\pm 2\sigma_2 = \pm 14.76$ & 95 & 95\% & 95\% \\\hline

    \end{tabular}
    \caption{Количество измерений в пределах $\pm\sigma$ и $\pm2\sigma$}
    \end{center}
    \end{table}

    \paragraph{}
    Как видим наши данные с довольно хорошей точностью соответствуют теории. Как видно из графика относительный разброс данных за 40с меньше чем за 10с. Подсчитаем какая разница между этими 2мя случаями.
    \[\frac{\sigma_1}{\bar{n}_1} \approx 26\%, \frac{\sigma_2}{\bar{n}_2} \approx 14\%\]
    \paragraph{}
    Как видим разница почти в 2 раза, что и следует от того факта что $\sigma \approx \sqrt{\bar{n}}$.
    \paragraph{}
    Для финального ответа подсчитаем ошибки средних величин. По теории
    \[\sigma_{\bar{n}_1} = \frac{\sigma_{1}}{N_1} \approx 0.17, \sigma_{\bar{n}_2} \approx 0.74\]
    \[\varepsilon_{\bar{n}_1} = \frac{\sigma_{\bar{n}_1}}{\bar{n}_1}\approx 1.3\%, \varepsilon_{\bar{n}_2}\approx 1.4\%\]
    Получаем финальный результат
    \[n_{t=10с}=13.06 \pm 0.17\]
    \[n_{t=40с}=52.23 \pm 0.74\]

    \newpage
    \begin{table}[H]
    \begin{center}
    \begin{tabular}{|c|r|r|r|r|r|r|r|r|r|r|}
    \hline
    {№ опыта} &   1 &   2 &   3 &   4 &   5 &   6 &   7 &   8 &   9 &  10 \\
    \hline
    0	&  23 &	 31	&  26 &	 20	&  24 &	 23	&  24 &	 29	&  36 &  25	\\
    10	&  29 &	 28	&  23 &  20	&  29 &  23	&  28 &  34	&  32 &	 27	\\
    20	&  23 &	 30	&  28 &	 21	&  23 &	 20	&  32 &	 25	&  21 &	 28	\\
    30	&  18 &	 27	&  21 &	 24	&  29 &	 18	&  23 &	 28	&  25 &	 28	\\
    40	&  32 &	 27	&  25 &	 33	&  24 &	 25	&  32 &	 36	&  26 &	 21	\\
    50	&  31 &	 34	&  22 &	 28	&  25 &	 31	&  25 &	 28	&  26 &	 22	\\
    60	&  34 &	 30	&  26 &	 24	&  28 &	 23	&  25 &	 27	&  13 &	 24	\\
    70	&  23 &	 28	&  30 &	 27	&  33 &	 33	&  21 &	 21	&  33 &	 24	\\
    80	&  33 &	 28	&  31 &	 24	&  34 &	 19	&  26 &	 32	&  22 &	 30	\\
    90	&  30 &	 25	&  29 &	 27	&  23 &	 28	&  29 &	 19	&  23 &	 19	\\
    100	&  27 &	 29	&  28 &	 33	&  29 &	 29	&  28 &	 24	&  26 &	 34	\\
    110	&  26 &	 26	&  21 &	 17	&  20 &	 24	&  27 &	 18	&  37 &	 30	\\
    120	&  30 &	 27	&  24 &	 21	&  29 &	 20	&  20 &	 24	&  19 &	 32	\\
    130	&  21 &	 17	&  26 &	 22	&  26 &	 24	&  25 &	 34	&  27 &	 28	\\
    140	&  27 &	 26	&  15 &	 26	&  20 &	 29	&  27 &	 19	&  23 &	 23	\\
    150	&  17 &	 21	&  16 &	 30	&  22 &	 34	&  40 &	 18	&  24 &	 33	\\
    160	&  32 &	 32	&  38 &	 34	&  25 &	 25	&  23 &	 17	&  30 &	 33	\\
    170	&  26 &	 28	&  22 &	 27	&  31 &	 23	&  34 &	 34	&  31 &	 31	\\
    180	&  26 &	 26	&  18 &	 33	&  23 &	 27	&  19 &	 27	&  19 &	 23	\\
    190	&  29 &	 27	&  24 &	 26	&  21 &	 37	&  24 &	 23	&  20 &	 19	\\
    \hline
    \end{tabular}
    \caption{Число срабатывании счетчика за 20с}
    \end{center}
    \end{table}


    \begin{table}[H]
    \begin{center}
    \begin{tabular}{|l|c|c|c|c|c|c|}\hline
    Число импульсов & 3 & 5 & 6 & 7 & 8 & 9 \\\hline
    Число случаев & 1 & 3 & 3 & 12 & 13 & 25 \\\hline
    Доля случаев & 0.0025 & 0.0075 & 0.0075 & 0.03 & 0.0325 & 0.0625 \\\hline
    \multicolumn{7}{c}{}\\\hline
    Число импульсов & 10 & 11 & 12 & 13 & 14 & 15 \\\hline
    Число случаев & 40 & 51 & 40 & 34 & 35 & 40 \\\hline
    Доля случаев & 0.1 & 0.1275 & 0.1 & 0.085 & 0.0875 & 0.1 \\\hline
    \multicolumn{7}{c}{}\\\hline
    Число импульсов &16 & 17 & 18 & 19 & 20 & 21 \\\hline
    Число случаев & 31 & 32 & 15 & 13 & 8 & 4 \\\hline
    Доля случаев & 0.0775 & 0.08 & 0.0375 & 0.0325 & 0.02 & 0.01 \\\hline
    \end{tabular}
    \caption{Данные для построения гистограммы для 10с}
    \end{center}
    \end{table}
    \newpage

    \begin{table}[H]
    \begin{center}
    \begin{tabular}{|c|r|r|r|r|r|r|r|r|r|r|}
    \hline
    {№ опыта} &   1 &   2 &   3 &   4 &   5 &   6 &   7 &   8 &   9 &  10 \\
    \hline
	0 &  54 &  46 &  47 &  53 &  61 &  57 &  43 &  52 &  62 &  59 \\
	10 &  53 &  49 &  43 &  57 &  49 &  45 &  45 &  47 &  51 &  53 \\
	20 &  59 &  58 &  49 &  68 &  47 &  65 &  50 &  56 &  53 &  48 \\
	30 &  64 &  50 &  51 &  52 &  37 &  51 &  57 &  66 &  42 &  57 \\	
	40 &  61 &  55 &  53 &  58 &  52 &  55 &  56 &  51 &  48 &  42 \\	
	50 &  56 &  61 &  58 &  52 &  60 &  52 &  38 &  44 &  45 &  67 \\	
	60 &  57 &  45 &  49 &  44 &  51 &  38 &  48 &  50 &  59 &  55 \\	
	70 &  53 &  41 &  49 &  46 &  46 &  38 &  46 &  56 &  58 &  57 \\ 	
	80 &  64 &  72 &  50 &  40 &  63 &  54 &  49 &  54 &  68 &  62 \\
	90 &  52 &  51 &  50 &  46 &  42 &  56 &  50 &  58 &  47 &  39 \\
    \hline
    \end{tabular}
    \caption{Число срабатывании счетчика за 40с}
    \end{center}
    \end{table}

    \begin{table}[H]
    \begin{center}
    \begin{tabular}{|l|c|c|c|c|c|c|c|c|c|c|c|}\hline
    Число импульсов & 37 & 38 & 39 & 40 & 41 & 42 & 43 & 44 & 45 & 46 & 47 \\\hline
    Число случаев & 1 & 3 & 1 & 1 & 1 & 3 & 2 & 2 & 4 & 5 & 4 \\\hline
    Доля случаев & 0.01 & 0.03 & 0.01 & 0.01 & 0.01 & 0.03 & 0.02 & 0.02 & 0.04 & 0.05 & 0.04 \\\hline
    \multicolumn{9}{c}{}\\\hline
    Число импульсов & 48 & 49 & 50 & 51 & 52 & 53 & 54 & 55 & 56 & 57 & 58 \\\hline
    Число случаев & 3 & 6 & 6 & 6 & 6 & 6 & 3 & 3 & 5 & 6 & 5 \\\hline
    Доля случаев & 0.03 & 0.06 & 0.06 & 0.06 & 0.06 & 0.06 & 0.03 & 0.03 & 0.05 & 0.06 & 0.05 \\\hline
    \multicolumn{9}{c}{}\\\hline
    Число импульсов & 59 & 60 & 61 & 62 & 63 & 64 & 65 & 66 & 67 & 68 & 72 \\\hline
    Число случаев & 3 & 1 & 3 & 2 & 1 & 2 & 1 & 1 & 1 & 2 & 1 \\\hline
    Доля случаев & 0.03 & 0.01 & 0.03 & 0.02 & 0.01 & 0.02 & 0.01 & 0.01 & 0.01 & 0.02 & 0.01 \\\hline
    \end{tabular}
    \caption{Данные для построения гистограммы для 40с}
    \end{center}
    \end{table}
    \newpage

    \begin{sidewaysfigure}
        \includegraphics[scale=0.45]{histogram.png}
        \caption{Гистограммы для $t=10с$ и $t=40с$}
    \end{sidewaysfigure}
\end{document}
