\documentclass[a4paper,12pt]{article} % добавить leqno в [] для нумерации слева
\usepackage[a4paper,top=1.3cm,bottom=2cm,left=1.5cm,right=1.5cm,marginparwidth=0.75cm]{geometry}
%%% Работа с русским языком
\usepackage{cmap}					% поиск в PDF
\usepackage{mathtext} 				% русские буквы в фомулах
\usepackage[T2A]{fontenc}			% кодировка
\usepackage[utf8]{inputenc}			% кодировка исходного текста
\usepackage[english,russian]{babel}	% локализация и переносы
\usepackage{multirow}

\usepackage{graphicx}

\usepackage{wrapfig}
\usepackage{tabularx}

\usepackage{hyperref}
\usepackage[rgb]{xcolor}
\hypersetup{
colorlinks=true,urlcolor=blue
}

%%% Дополнительная работа с математикой
\usepackage{amsmath,amsfonts,amssymb,amsthm,mathtools} % AMS
\usepackage{icomma} % "Умная" запятая: $0,2$ --- число, $0, 2$ --- перечисление

%% Номера формул
\mathtoolsset{showonlyrefs=true} % Показывать номера только у тех формул, на которые есть \eqref{} в тексте.

%% Шрифты
\usepackage{euscript}	 % Шрифт Евклид
\usepackage{mathrsfs} % Красивый матшрифт

%% Свои команды
\DeclareMathOperator{\sgn}{\mathop{sgn}}

%% Перенос знаков в формулах (по Львовскому)
\newcommand*{\hm}[1]{#1\nobreak\discretionary{}
{\hbox{$\mathsurround=0pt #1$}}{}}

\date{\today}

\begin{document}

\begin{titlepage}
	\begin{center}
		{\large МОСКОВСКИЙ ФИЗИКО-ТЕХНИЧЕСКИЙ ИНСТИТУТ (НАЦИОНАЛЬНЫЙ ИССЛЕДОВАТЕЛЬСКИЙ УНИВЕРСИТЕТ)}
	\end{center}
	\begin{center}
		{\large Физтех-школа радиотехники и компьютерных технологий}
	\end{center}
	
	
	\vspace{4.5cm}
	{\huge
		\begin{center}
			{\bf Отчёт о выполнении лабораторной работы 1.4.1}\\
			Изучение физического маятника
		\end{center}
	}
	\vspace{2cm}
	\begin{flushright}
		{\LARGE Автор:\\ Уткин Павел \\
			\vspace{0.2cm}
			Б01-308}
	\end{flushright}
	\vspace{8cm}
	\begin{center}
		Долгопрудный 2023
	\end{center}
\end{titlepage}

\section{Введение}

\textbf{Цель работы:} исследовать зависимость периода колебаний физического маятника от момента его инерции.\\
\textbf{В работе используются:} физический маятник (однородный стальной стержень), опорная призма, счётчик числа колебаний, линейка, секундомер.

\section{Теоретические сведения}

\begin{wrapfigure}{l}{6cm}
	\includegraphics[width=1\linewidth]{ustanovka.png}
	\caption{Физический маятник}\label{risunok}
\end{wrapfigure}

Физическим маятником называют любое твердое тело, которое под действием силы тяжести может свободно качаться вокруг неподвижной горизонтальной оси. Движение маятника ­описывается уравнением

\begin{equation}
I\frac{d^2\varphi}{dt^2}=M,
\label{osnova}
\end{equation}

\noindent где $ I $ -- момент инерции маятника, $ \varphi $ -- угол отклонения маятника от положения равновесия, $ t $ - время, $ М $ - момент сил, действующих на маятник.

В данной работе в качестве физического маятника (рис.~ \ref{risunok}) используется однородный стальной стержень длиной $ l $. На стержне закрепляется опорная призма, острое ребро которой является осью качания маятника. Призму можно перемещать вдоль стержня, меняя таким образом расстояние $ OC $ от точки опоры маятника до его центра масс. Пусть это расстояние равно $ a $. Тогда по теореме Гюйгенса-Штейнера момент инерции маятника

\begin{equation}
I=\frac{ml^2}{12}+ma^2,
\end{equation}

\noindent где $ m $ -- масса маятника. Момент силы тяжести, действующий на маятник, 

\begin{equation}
M=-mga\sin\varphi.
\end{equation}

\noindent Если угол $ \varphi $ мал, то $ \sin\varphi\approx\varphi $, так что

\begin{equation}
M\approx-mga\varphi
\end{equation}

\noindent В исправной установке маятник совершает несколько сот колебаний без заметного затухания. Поэтому моментом силы трения в первом приближении можно пренебречь. Подставляя выражение для $ I $ и $ M $ в \eqref{osnova}, получим уравнение

\begin{equation}
\ddot{\varphi}+\omega^2\varphi=0,
\label{phi}
\end{equation}

\noindent где

\begin{equation}
\omega^2=\frac{ga}{a^2+\frac{l^2}{12}}.
\end{equation}

Тогда период колебаний равен

\begin{equation}\label{period}
T=\frac{2\pi}{\omega}=2\pi\sqrt{\frac{a^2+\frac{l^2}{12}}{ag}}
\end{equation}

Таким образом, период малых колебаний не зависит ни от начальной фазы, ни от амплитуды колебаний. Это утверждение (изохорность) справедливо для колебаний, подчиняющихся уравнению \eqref{phi}. Движение маятника описывается по этой формуле только для малых углов $ \varphi $.

\medskip

Период колебаний математического маятника определяется формулой
\begin{equation}
T'=2\pi\sqrt{\frac{l'}{g}},
\end{equation}
где $ l' $ -- длина математического маятника. Поэтому величину
\begin{equation}\label{prived}
l_\text{пр}=a+\frac{l^2}{12a}
\end{equation}
называют приведённой длиной математического маятника. Поэтому точку $ O' $ (см. рис. \ref{risunok}), отстоящую от точки опоры на расстояние $ l_\text{пр} $, называют центром качания физического маятника. Точка опоры и центр качания маятника обратимы, т.е. при качании маятника вокруг  точки $ O' $ период будет таким же, как и при качании вокруг точки $ O $.

\section{Оборудование и экспериментальные погрешности}

\textbf{Секундомер:} $ \Delta_c = 0,01 \text{ с}$\\
\textbf{Линейка:} $ \Delta_\text{лин} = 0,05 \text{ см}$

\section{Результаты измерений и обработка данных}
\subsection{Определение диапазона амплитуд, в пределах которых период колебаний маятника можно считать не зависящим от амплитуды}\label{formuli}

Определим диапазон амплитуд, в пределах которым колебания можно считать не зависящими от начальной фазы. Для этого отклоняем маятник на угол $ \varphi \approx 10^\circ  $ и измеряем период колебаний. Затем уменьшаем угол отклонения в 2 раза и повторяем измерения. Результаты приведены в таблице \ref{tab1}.

\begin{table}[h!]
	\begin{center}
		\begin{tabular}{|c|c|c|c|c|c|c|c|}
		\hline
		№ & $ \varphi $ & $ T_\text{общ} $, с & $ N_\text{изм} $ & $ T $, с   & $ T_\text{ср} $, с     & $ \sigma $, с     & $\varepsilon$, $ \% $       \\ \hline
		1 & 10 & 153,92    & 100        & 1,5392 &          &          &             \\ \cline{1-5}
		2 & 10 & 153,87    & 100        & 1,5387 & 1,5389 & 0,0002 & 0,0130 \\ \cline{1-5}
		3 & 10 & 153,89    & 100        & 1,5389 &          &          &             \\ \hline
		4 & 5  & 154,08    & 100        & 1,5408 &          &          &             \\ \cline{1-5}
		5 & 5  & 154,12    & 100        & 1,5412 & 1,5411 & 0,0002 & 0,0129 \\ \cline{1-5}
		6 & 5  & 154,13    & 100        & 1,5413 &          &          &             \\ \hline
	\end{tabular}
	\end{center}
	\caption{Результаты измерения периода колебаний в зависимости от начального угла}
\label{tab1}
\end{table}

Время одного колебания для отдельного эксперимента рассчитываем по формуле:

\begin{equation}
T=\frac{T_\text{общ}}{N_\text{изм}},
\end{equation}
где $ T_\text{общ} $ -- время $ N_\text{изм} $ колебаний.

Среднее значение периода колебаний для одной серии опытов рассчитываем по следующей формуле:

\begin{equation}
T_\text{ср}=\frac{1}{N_\text{изм}}\sum_{i=1}^{N_\text{изм}}T_i.
\end{equation}

Случайная погрешность определения $ T_\text{ср} $ вычисляется по формуле:

\begin{equation}
\sigma_\text{сл}=\sqrt{\frac{1}{N_\text{изм}\left( N_\text{изм} - 1 \right)}\sum_{i=1}^{N_\text{изм}}\left( T_\text{ср} - T_i \right)^2 }.
\end{equation}

Полная погрешность измерения периода колебаний в одной серии опытов определяется следующим соотношением:

\begin{equation}
\sigma = \sqrt{\Delta_\text{с}^2+\sigma_\text{сл}^2}.
\end{equation}

Также можем рассчитать относительную погрешность:

\begin{equation}
\varepsilon = \frac{\sigma}{T_\text{ср}}.
\end{equation}

Исходя из измерений и пользуясь представленными выше формулами, получаем, что\\ $  T_{10^\circ} =\left( 1,5389 \pm 0,0002\right)  $ с и $ T_{5^\circ} =\left(  1,5411 \pm 0,0002\right)  $ с. Следовательно, полученные значения периода колебаний маятника совпадают с хорошей точностью. Поэтому в дальнейшем будем считать колебания не зависящими от начальной фазы при отклонении на угол не более $ \varphi_0 \approx 5^\circ$.

\subsection{Исследуем заваисимость преиода колебаний от расстояния между точкой опоры и центром масс}

Для вычисления ускорения свободного падения и длины стержня будем исследовать зависимость периода колебаний $ T $ от расстояния $ a $ между точкой опоры и центром масс. Результаты проведённых измерений представлены в таблице \ref{tab2}.

\begin{table}[h!!]
	\begin{center}
		\begin{tabular}{|c|c|c|c|c|c|c|c|c|c|}
		\hline
		№ &
		$ a $, см &
		$ T_\text{общ} $, с &
		$ N_\text{изм} $ &
		$ T $, с &
		$ T_\text{ср} $, с &
		$ \Delta $, с &
		$ \sigma_\text{сл} $, с &
		$ \sigma_T$, с &
		$ \varepsilon $, $ \% $ \\ \hline
		1 &
		7,3 &
		212,47 &
		100 &
		2,1247 &
		\multirow{2}{*}{2,1240} &
		\multirow{2}{*}{0,0001} &
		\multirow{2}{*}{$7,04 \cdot 10^{-5}$} &
		\multirow{2}{*}{0,000122} &
		\multirow{2}{*}{0,0058} \\ \cline{1-5}
		2 &
		7,3 &
		212,33 &
		100 &
		2,1233 &
		&
		&
		&
		&
		\\ \hline
		3 &
		10,4 &
		189,51 &
		100 &
		1,8951 &
		\multirow{2}{*}{1,8959} &
		\multirow{2}{*}{0,0001} &
		\multirow{2}{*}{$8,04 \cdot 10^{-5}$} &
		\multirow{2}{*}{0,0001} &
		\multirow{2}{*}{0,0067} \\ \cline{1-5}
		4 &
		10,4 &
		189,67 &
		100 &
		1,8967 &
		&
		&
		&
		&
		\\ \hline
		5 &
		13,3 &
		175,34 &
		100 &
		1,7534 &
		\multirow{2}{*}{1,7528} &
		\multirow{2}{*}{0,0001} &
		\multirow{2}{*}{$5,53 \cdot 10^{-5}$} &
		\multirow{2}{*}{0,0001} &
		\multirow{2}{*}{0,0065} \\ \cline{1-5}
		6 &
		13,3 &
		175,23 &
		100 &
		1,7523 &
		&
		&
		&
		&
		\\ \hline
		7 &
		16,4 &
		165,72 &
		100 &
		1,6572 &
		\multirow{2}{*}{1,6576} &
		\multirow{2}{*}{0,0001} &
		\multirow{2}{*}{$5,53 \cdot 10^{-5}$} &
		\multirow{2}{*}{0,0001} &
		\multirow{2}{*}{0,0065} \\ \cline{1-5}
		8 &
		16,4 &
		165,81 &
		100 &
		1,6581 &
		&
		&
		&
		&
		\\ \hline
		9 &
		18,3 &
		160,16 &
		100 &
		1,6016 &
		\multirow{2}{*}{1,6021} &
		\multirow{2}{*}{0,0001} &
		\multirow{2}{*}{$5,53 \cdot 10^{-5}$} &
		\multirow{2}{*}{0,0001} &
		\multirow{2}{*}{0,0071} \\ \cline{1-5}
		10 &
		18,3 &
		162,27 &
		100 &
		1,6027 &
		&
		&
		&
		&
		\\ \hline
		11 &
		24 &
		154,08 &
		100 &
		1,5408 &
		\multirow{2}{*}{1,5410} &
		\multirow{2}{*}{0,0001} &
		\multirow{2}{*}{$2,01 \cdot 10^{-5}$} &
		\multirow{2}{*}{0,0001} &
		\multirow{2}{*}{0,0066} \\ \cline{1-5}
		12 &
		24 &
		154,12 &
		100 &
		1,5412 &
		&
		&
		&
		&
		\\ \hline
		13 &
		26,5 &
		153,02 &
		100 &
		1,5302 &
		\multirow{2}{*}{1,5298} &
		\multirow{2}{*}{0,0001} &
		\multirow{2}{*}{$3,52 \cdot 10^{-5}$} &
		\multirow{2}{*}{0,0001} &
		\multirow{2}{*}{0,0069} \\ \cline{1-5}
		14 &
		26,5 &
		152,95 &
		100 &
		1,5295 &
		&
		&
		&
		&
		\\ \hline
	\end{tabular}
	\end{center}
\caption{Измерение периода колебаний маятника $ T $ в зависимости от расстояния $ a $}
\label{tab2}
\end{table}

По полученным данным вычисляем $ T $, $ T_\text{ср} $, $ \Delta $, $ \sigma_\text{сл}$, $ \sigma $ и $ \varepsilon $ по формулам представленным в разделе \ref{formuli}.

Используя формулу для периода физического маятника \eqref{period} получаем следующее соотношение:

\begin{equation}
T^2a=\frac{4\pi^2}{g}a^2+\frac{\pi^2l^2}{3g}.
\end{equation}
Отсюда можно сделать вывод о том, что $ T^2a $ линейно зависит от $ a^2 $, поэтому это зависимость можно представить в виде

\begin{equation}
T^2a=ka^2+b,
\end{equation}
где
\begin{equation}\label{koef}
k=\frac{4\pi^2}{g}  \text{ и }  b = \frac{\pi^2l^2}{3g}.
\end{equation}

Найдём эти коэффициента. Для удобства перенесём все необходимые данные в таблицу \ref{tab3}.


\begin{table}[h!]
    \begin{center}
	\begin{tabular}{|c|c|c|c|c|c|c|c|}
		\hline
		$ a $, см           & 7,3      & 10,4     & 13,3     & 16,4     & 18,3     & 24     & 26,5     \\ \hline
		$ a^2 $, $ \text{см}^2 $  & 53,29     & 108,16    & 176,89    & 268,96    & 334,89    & 576    & 702,25   \\ \hline
		$ \sigma_{a^2}$,$ \text{см}^2  $         & 0,73    & 1,04      & 1,33    & 1,64      & 1,83    & 2,4    & 2,65    \\ \hline
		$ \varepsilon_{a^2}$ , $ \% $         & 1,37    & 0,96      & 0,75    & 0,61      & 0,55    & 0,42    & 0,38    \\ \hline
		$ T $, с           & 2,124  & 1,896  & 1,753  & 1,658  & 1,602  & 1,541  & 1,530  \\ \hline
		$ T^2 $, $ \text{с}^2 $   & 4,511  & 3,594  & 3,072  & 2,748  & 2,567  & 2,375  & 2,340    \\ \hline
		$ aT^2 $, $ \text{см} \cdot \text{с}^2 $ & 32,933 & 37.382 & 40,864 & 45,064 & 46,974 & 56,922 & 62,022 \\ \hline
		$ \sigma_{aT^2} $, $ \text{см} \cdot \text{с}^2  $       & 0,226  & 0,182  & 0,154  & 0,138  & 0,129  & 0,119  & 0,117   \\ \hline
		$ \varepsilon_{aT^2} $, $ \% $       & 0,69  & 0,48  & 0,38  & 0,31  & 0,27  & 0,21  & 0,19  \\ \hline
	\end{tabular}
    \end{center}
\caption{Значения $ aT^2 $ и $ a^2 $ и их погрешности}
\label{tab3}
\end{table}

% График зависимости $ aT^2 $ от $ a^2 $ представлен на рисунке \ref{graph}.

% \begin{figure}[h!]
% 	\includegraphics[scale=0.38]{graph_file.jpeg}
% 	\caption{Зависимость $ aT^2 $ от $ a^2 $}
% 	\label{graph}
% \end{figure}

Погрешность расчёта $ a^2 $ найдём по следующей формуле:

\begin{equation}
\sigma_{a^2}=2a^2\frac{\Delta a}{a}=2a\Delta a,
\end{equation}
где $ \Delta a = \Delta_\text{с} = 0,05$ см.

Погрешность вычисления $ aT^2 $ можно найти по формуле:

\begin{equation}
\sigma_{aT^2} = \sqrt{\left(  \frac{\Delta a}{a} \right)^2 + \left( 2\frac{\Delta T}{T} \right)^2 },
\end{equation}
где $ \Delta T = \sigma_T $ из таблицы \ref{tab2}.

Для вычисления коэффициентов $ k $ и $ b $ из \eqref{koef} воспользуемся методом наименьших квадратов:

\begin{equation}
k=\frac{\langle xy\rangle-\langle x\rangle \langle y\rangle}{\langle x^2\rangle - \langle x\rangle^2}\approx 0,0426\text{ }\frac{\text{с}^2}{\text{см}},
\end{equation}

\begin{equation}
b=\langle y \rangle -k\langle x \rangle\approx 32,59\text{ }\text{см}\cdot\text{с}^2,
\end{equation}
где $ x=a^2 $, $ y=aT^2 $.

Случайные погрешности вычисления $ k $ и $ b $ можно найти по следующим формулам:

\begin{equation}
\sigma_k^\text{сл}=\sqrt{\frac{1}{N-2}\left(\frac{\langle y^2 \rangle - \langle y \rangle^2}{\langle x^2 \rangle - \langle x \rangle^2} - k^2 \right) } \approx 0,0018 \text{ }\frac{\text{с}^2}{\text{см}},
\end{equation}

\begin{equation}
\sigma_b^\text{сл}= \sigma_k^\text{сл} \sqrt{\langle x^2 \rangle - \langle x \rangle^2} \approx 0,4134 \text{ }\text{см}\cdot\text{с}^2.
\end{equation}

Систематическая погрешность вычисления коэффициентов определяется следующим соотношением:

\begin{equation}
\sigma^\text{сист}_k = k\sqrt{\left( \varepsilon_{aT^2} \right)^2 + \left( \varepsilon_{a^2} \right)^2 } \approx 0,0034 \text{ }\frac{\text{с}^2}{\text{см}},
\end{equation}

\begin{equation}
\sigma^\text{сист}_b = b\sqrt{\left( \varepsilon_{aT^2} \right)^2 + \left( \varepsilon_{a^2} \right)^2 } \approx  2,62 \text{ }\text{см}\cdot\text{с}^2.
\end{equation}

Тогда полную погрешность вычисления коэффициентов подсчитываем по следующей формуле:

\begin{equation}
\sigma_k = \sqrt{\left( \sigma_k^\text{сл} \right)^2 + \left( \sigma_k^\text{сист} \right)^2 } \approx 0,0038 \text{ }\frac{\text{с}^2}{\text{см}},
\end{equation}

\begin{equation}
\sigma_b = \sqrt{\left( \sigma_b^\text{сл} \right)^2 + \left( \sigma_b^\text{сист} \right)^2 } \approx 2,65 \text{ }\text{см}\cdot\text{с}^2.
\end{equation}

Таким образом, получаем:
\begin{itemize}
	\item $ k = \left( 0,0426\pm0,0038\right)  \text{ }\frac{\text{с}^2}{\text{см}} $, $ \varepsilon_k = 8,92 \% $
	\item $ b = \left( 32,59\pm2,65\right)  \text{ }\text{см}\cdot\text{с}^2 $, $ \varepsilon_b = 8,13 \% $
\end{itemize}

Учитывая формулу \eqref{koef}, вычисляем $ g $ и $ l $:

\begin{equation}
g = \frac{4\pi^2}{k} \approx 9,28 \text{ }\frac{\text{м}}{\text{с}^2},
\end{equation}

\begin{equation}
\sigma_g = g\cdot\varepsilon_k \approx 0,83 \text{ }\frac{\text{м}}{\text{с}^2},
\end{equation}

\begin{equation}
l=\sqrt{\frac{3bg}{\pi^2}}\approx 95,88 \text{ см},
\end{equation}

\begin{equation}
\sigma_l = l\sqrt{\left( \frac{1}{2}\varepsilon_b \right)^2 + \left( \frac{1}{2}\varepsilon_g \right)^2 }\approx 5,8 \text{ см}.
\end{equation}

В итоге имеем следующие результаты:

\begin{itemize}
	\item \underline{$ g = \left( 9,28\pm0,83\right) \frac{\text{м}}{\text{с}^2} $, $ \varepsilon_g=8,9\% $}
	\item \underline{$ l = \left( 95,88\pm5,8\right) \text{см} $, $ \varepsilon_l=6,05\% $}
\end{itemize}

\subsection{Проверка обратимости точки подвеса и центра качания физического маятника}

Для $ a_1 = 25 $ см вычислим $ l_\text{пр} $:

\begin{equation}
l_\text{пр}=a_1+\frac{l^2}{12a_1}\approx 58,3 \text{ см}.
\end{equation}
Следовательно расстояние от центра масс до центра качания $ a_2 $ вычисляем следующим образом:
\begin{equation}
a_2=\left|a_1-l_\text{пр}\right|\approx 33 \text{ см}
\end{equation}

\begin{table}[h!]
	\begin{center}
		\begin{tabular}{|c|c|c|c|c|c|c|c|}
			\hline
			№ & $ a $, см & $ T_\text{общ} $, с & $ N_\text{изм} $ & $ T_1 $, с  & $ T_\text{ср} $, с               & $ \sigma_T $, с                & $ \varepsilon,\% $                \\ \hline
			1 & 25    & 138,19    & 90         & 1,5354 & \multirow{2}{*}{1,5354} & \multirow{2}{*}{0,0001} & \multirow{2}{*}{$ 0,007$} \\ \cline{1-5}
			2 & 25    & 138,18    & 90         & 1,5353 &                         &                         &                              \\ \hline \hline
			1 & 33    & 153,31    & 100        & 1,5331 & \multirow{2}{*}{1,5336} & \multirow{2}{*}{0,0001} & \multirow{2}{*}{$ 0,008$} \\ \cline{1-5}
			2 & 33    & 153,44    & 100        & 1,5344 &                         &                         &                              \\ \hline
		\end{tabular}
	\end{center}
\caption{Измерение периода колебаний физического маятника для $ a_1=25 $ см и $ a_2=33 $ см}\label{tab5}
\end{table}

Проведём измерение периода колебаний физического маятника для $ a_1=25 $ см и $ a_2=33 $ см. Результаты измерений представлены в таблице \ref{tab5}.\\

Получаем, что
\begin{itemize}
	\item \underline{$ T_{\text{ср}1} = \left( 1,5354\pm0,0001 \right) $ с}
	\item \underline{$ T_{\text{ср}2} = \left( 1,5336\pm0,0001 \right) $ с}
\end{itemize}
Следовательно, $ T_{\text{ср}1} \approx T_{\text{ср}2} $ и точка подвеса и центр качания физического маятника обратимы.

\section{Выводы и обсуждение результатов}

В ходе работы были получены следующие величины:
\begin{itemize}
	\item \underline{$ g = \left( 9,28\pm0,83\right) \frac{\text{м}}{\text{с}^2} $, $ \varepsilon_g=8,9\% $}
	\item \underline{$ l = \left( 95,88\pm5,8\right) \text{см} $, $ \varepsilon_l=6,05\% $}
\end{itemize}
Они были получены с не достаточно хорошей точностью, что говорит о погрешностях при измерениях.\\
Также, были экспериментально проверена теория о приведённой длине физического маятника и теория об обратимости точки подвеса и центра качания.
Точность полученных результатов можно повысить, если исключить ошибку при фиксации периода колебаний маятника, которая существует в силу неидеальной реакции экспериментатора. В не меньшей степени на результат повлияло трение в точке подвеса. Также свою погрешность вносит неточность определения расстояния от точки опоры до центра масс стержня.










\end{document}
